\newpage \documentclass[a4paper,10pt]{article}
\usepackage[spanish]{babel}
\usepackage[T1]{fontenc}
\usepackage[utf8]{inputenc}
\usepackage{anysize} % Soporte para el comando \marginsize
\usepackage{listings}
\usepackage{formular}
\usepackage[pdftex]{graphicx}
\usepackage[colorlinks,linkcolor=black,citecolor=black, urlcolor=black]{hyperref}
\usepackage{appendix}
\usepackage{float}
\renewcommand{\appendix}{%
  \setcounter{section}{0}%
}

%
\DeclareGraphicsExtensions{.pdf,.png,.jpg}

\usepackage{color}
\definecolor{gray97}{gray}{.97}
\definecolor{gray75}{gray}{.75}
\definecolor{gray45}{gray}{.45}

\lstset{ frame=Ltb,
  framerule=0pt,
  aboveskip=0.5cm,
  framextopmargin=3pt,
  framexbottommargin=3pt,
  framexleftmargin=0.4cm,
  framesep=0pt,
  rulesep=.4pt,
  backgroundcolor=\color{gray97},
  rulesepcolor=\color{black},
  %
  stringstyle=\ttfamily,
  showstringspaces = false,
  basicstyle=\small\ttfamily,
  commentstyle=\color{gray45},
  keywordstyle=\bfseries,
  %
  numbers=left,
  numbersep=15pt,
  numberstyle=\tiny,
  numberfirstline = false,
  breaklines=true,
}

% minimizar fragmentado de listados
\lstnewenvironment{listing}[1][]
{\lstset{#1}\pagebreak[0]}{\pagebreak[0]}

\lstdefinestyle{consola}
{basicstyle=\scriptsize\bf\ttfamily,
  backgroundcolor=\color{gray75},
}

\renewcommand*\lstlistingname{Listado}

\marginsize{3cm}{3cm}{2.5cm}{2.5cm}


\title{Servicios de Computación de Altas Prestaciones y Disponibilidad \\
MPI}
\author{Daniel Aguado Araujo}

\begin{document}

\maketitle

\newpage

\tableofcontents

\newpage

\section{Enunciado}


Conocida una función f(x) mediante una secuencia de N pares $(x_{i},f(x_{i}))$ , tal que $x_{i - 1} < x_{i}$.\\

Se conoce como:\\

\textbf{Suma de Riemann por la izquierda.}\\

$S_{izquierda} = \sum_{i=1}^{N -1} f(x_{i - 1}) * (x_{i} - x_{i - 1})$\\

\textbf{Suma de Riemann por la derecha.}\\

$S_{derecha} = \sum_{i=1}^{N -1} f(x_{i}) * (x_{i} - x_{i - 1})$\\

\textbf{Suma de Riemann trapezoidal.}\\

$S_{izquierda} = \sum_{i=1}^{N -1} \frac{f(x_{i - 1}) + f(x_{i})}{2} * (x_{i} - x_{i - 1})$\\
\\


Desarrollar un programa en C con MPI que calcule de forma paralela, en un entorno con P procesadores, los valores $S_{izquierda}$, $S_{derecha}$ y $S_{trapezoidal}$ de una función definida por puntos almacenada en un fichero secuencial \textit{data.dat} de N puntos, y muestre el tiempo que ha tardado en ejecutarse. \\

El fichero almacena los puntos por parejas [$x_{0},f(x_{0}),x_{1},f(x_{1}), ..., ,x_{N - 1},f(x_{N - 1})$]. \\

Todos los valores son de tipo \textit{double}.\\

\textbf{Nota}\\

Realizar la implementación mediante un proceso inicial que lee el fichero y distribuye los datos entre los demás procesos, para finalmente mostrar el resultado.\\

El tiempo de ejecución a calcular, será el transcurrido a partir de recibir todos los procesos sus datos.\\

\subsection{Entrega}

\begin{itemize}
\item \textbf{Documento Teoria.pdf} En este documento se explicita los pasos necesarios para realizar el algoritmos paralelo: división, comunicaciones, agrupación y asignación. También se incluirá una breve descripción del algoritmo y su implementación.

\item \textbf{Fichero fuente Riemann.c}  Este fichero deberá de compilarse \textit{mpicc -o nombre\_del\_ejecutable Riemann.c}  sin errores para poder ejecutar sin error el siguiente comando: \textit{mpirun -np numero\_procesadores nombre\_del\_ejecutable tamaño\_del\_problema Fichero.dat}. El resultado deberá ser:
\begin{itemize}
\item Sizq = Valor
\item  Sder = Valor
\item  Strap = Valor
\item  Tiempo máximo de ejecución = Valor segundos.
\end{itemize}

\end{itemize}


%\begin{figure}[!h]
%\begin{center}
%\includegraphics[width=1\textwidth]{path.png}
%\caption{detail}
%\label{fig:label}
%\end{center}
%\end{figure}









































\end{document}
